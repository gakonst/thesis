\section{History}
In 2009 Satoshi Nakamoto published the Bitcoin whitepaper~\cite{bitcoin}. There, Nakamoto describes `a purely peer-to-peer version of electronic cash would allow online payments to be sent directly from one party to another without going through a financial institution.'
In the beginning, Bitcoin was primarily used for fast and low-cost financial transactions. It was soon realized that its uses could be extended to more than just transfering value from A to B. The concept of colored coins, \cite{colored} was introduced, where users were able to embed extra data on a bitcoin which resulted in coins that could represent represent ownership over a land title or a domain name. In 2015, Vitalik Buterin authored the Ethereum Whitepaper \cite{vitalik} which was an alternative cryptocurrency to Bitcoin that enabled the creation of \textit{smart Contracts}. smart Contracts as a term was first introduced by Nick Szabo in 1996 \cite{szabo} as a model for verified trustless computation. The Ethereum Network acts as a world computer and smart contracts are code that gets executed trustlessly on every node that is part of the network. 

\section{Problem Statement}
The problem this Master Thesis solves is:
How an entity can manage the energy consumed by a complex system of energy meters.
The  system should be able to bill and perform accounting on the metering data, based on a pre-specified accounting model which can be changed at runtime.
The system must be transparent, distributed, decentralized, easy-to-use and secure.
Anyone in the network should be able to verify the validity transactions.
It also needs to be scalable at reasonable cost.
%TODO Rephrase

\section{Scope}
The Master Thesis explores the fundamental terms needed to understand blockchain terminology. The contributions to scalability are limited to optimizing smart-contracts with respect to not stressing the network. Larger scale scalability solutions such as alternative consensus algorithms, payment channels or sidechains are out of scope. On security, the industry's best practices are applied, while also utilizing tools used by smart contract auditing firms, along with a proprietary tool that was provided for furter analysis.

\section{Outline}
Chapter X describes Y