% who why what when how 
\abstract{
    
In 2009 Satoshi Nakamoto published the Bitcoin whitepaper~\cite{bitcoin} where he described `a purely peer-to-peer version of electronic cash' which `would allow online payments to be sent directly from one party to another without going through a financial institution'. 

Bitcoin was primarily used for fast and low-cost financial transactions which were routed without any bank interference. It was soon realized that its underlying technology, blockchain, could be used for more than transferring financial value. A blockchain is a database that can be shared between a group of non-trusting individuals, without needing a central party to maintain the state of the database. The data in a blockchain is transparent and secured via cryptography. As more advanced blockchain platforms were built on top of Bitcoin~\cite{colored}, in the end of 2014 a blockchain platform which was capable of executing smart contracts was released, called Ethereum~\cite{vitalik}. A smart contract is software that is executed on a blockchain and can be used as a framework for secure and trustless computation. 

We leverage the power of the Ethereum blockchain and smart contracts to create a system that can transparently and securely perform metering of energy as well as perform accounting for the consumed energy based on specific business logic. The advantages and disadvantages of smart contracts are explored. Due to the distributed nature of blockchains there are challenges towards achieving scalability and transaction throughput comparable to traditional centralized payment processors or server architectures. Smart contract security has been a pressing issue as large financial amounts have been stolen from smart contracts.

Past literature on current scalability and security issues of smart contracts is studied. Contributions are made on scalability by proposing a method to make data storage on smart contracts more efficient. On security we utilize and augment the functionality of an auditing tool in order to analyze and identify vulnerabilities in smart contracts. Finally, we apply the gained insight and techniques on the metering-billing use case in order to enhance its viability and robustness in production.

}
