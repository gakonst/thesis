% who why what when how 
\abstract{
    
We leverage the power of the Ethereum blockchain and smart contracts to create a system that can transparently and securely perform metering of energy as well as perform accounting for the consumed energy based on specific business logic. The advantages and disadvantages of smart contracts are explored. Due to the distributed nature of blockchains there are challenges towards achieving scalability and high transaction throughput, which traditional centralized payment processors or server architectures provide. Smart contract security has been a pressing issue as large financial amounts have been stolen from smart contracts.

Past literature on current scalability and security issues of smart contracts is studied. Contributions are made on scalability by proposing a method to make data storage on smart contracts more efficient. On security we utilize and augment the functionality of an auditing tool in order to analyze and identify vulnerabilities in smart contracts. We apply the gained insight and techniques on the metering-billing use case in order to enhance its viability and robustness in production. Finally, we evaluate the effectiveness of the proposed scalability techniques and security best-practices on the written smart contracts.

}
