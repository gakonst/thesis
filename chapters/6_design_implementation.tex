\chapter{Design and Implementation}
We proceed to discuss the business logic of the use-case as described in \ref{usecase} and then implement it. We utilize the technique from \ref{method3} to optimize our smart contracts for gas efficiency and utilize Slither from \ref{slither} and the learned best-practices to ensure that the developed smart contracts are robust. Due to the intellectual property of Honda R\&D, all testing and verification of the contracts' functionality was done in a private in-house testnet.

\section{Business Logic}

\subsection{Smart Meters and Rooms}

A smart meter must be able to keep track of the current reading and timestamp of the reading as well as the last reading and timestamp in order to calculate the difference of the two. It also has a unique identifier which is used to retrieve it in the smart contract.

A company building is split into rooms. Each smart meter contributes to a room's consumption with a real coefficient, according to Equation \ref{roomcost}
\begin{equation}\label{roomcost}
R = C * M
\end{equation}
\noindent where
\begin{description}
\item  \[C = 
\begin{bmatrix}
    c_{11} & \cdots & c_{1M} \\
    \vdots & \ddots & \vdots \\ 
    \vdots & \ddots & \vdots \\
    c_{N1} & \cdots & c_{NM}
  \end{bmatrix}\] ($c_{ij}$ is the coefficient of the $jth$ meter for the $ith$ room)

\item  and \[M = 
\begin{bmatrix}
    m_{1} & \cdots & m_{M}
\end{bmatrix}^T\] ($m_{i}$ is the kilowatthour reading of the $ith$ meter)
\end{description}

The coefficients are calculated throuhg an internal partner, INSERT DETAILS ON EASD.

\subsection{Cost Centers and Billing}

Rooms are grouped together in a structure called \textit{Cost Center} which does X. A room can belong only to one cost center. During the accounting stage, the accountant can retrieve the difference in energy consumption since the last clearing period and thus


\section{Smart Contracts}

In this section we go over the implementation and the rationale of each developed smart contract. We explain the inner workings and provide tests of their functionality. A thorough walkthrough on how they interact with each other can be found in \ref{scresults}
\subsection{Contract Registry}
Upgradable logic, call smart contract by name.

\subsection{Meter Management}
Meter manager utilizes \ref{method3}. Each meter has its own ID. We use the pattern. Deleting a meter sets the active status to false. We iterate over the array of meters. There are software engineering patterns \cite{crud} that allow more proper usage, however they cost a lot more gas. 

\subsection{Cost - Profit Management}
We follow the same pattern as with meters for storing cost centers. We define 
Follow busienss logic

\subsection{Access Control}
Defining a proper access control policy is very important as discussed in Section \ref{security}. It is common to find Access Control Lists (ACL) in enterprise environments which allow access to resources only to selected participants. This does not exist by default in smart contracts. The Aragon Project\footnote{Project aimed at creating DAOs} provides an ACL contract, however it was not used in the final version due to the complexity it introduced to our code\footnote{Aragon's contracts are architected towards creating fully upgradable DAOs, which would introduce considerable overheads and complexity to our code}. Instead, the DSAUTH pattern is used.

\section{Monitoring Server}
Could be implemented without monitoring server if each meter was smarter. 
Explain monitoring server

\subsection{REST API} 
Explain rest api usage 

\subsection{Python Client}
Explain python implementation of rest api

\subsection{web3.py interaction}
Explain how web3.py interacts with monitoring server and sends data to Smart Contracts

\section{Network Topology}
% INSERT GRAPH OF PULLING DATA FROM MONITORING SERVER TO CLIENT, CLIENT SIMULATES THE METERS AND SENDS THE DATA SIGNED WITH THEIR PRIVATE KEYS. TRANSACTIONS GET EXECUTED IN A 3-NODE POA BLOCKCHAIN. EXPLAIN DOCKER HERE. 
