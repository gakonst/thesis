\chapter{Introduction}

\section{Problem Statement}
We aim to answer the following research questions in this thesis:

\begin{enumerate}
    \item How can the scalability of smart contracts be improved?
    \item How can the security of smart contracts be improved?
    \item How can a system that is able to measure and bill the energy consumed by a set of energy meters be modeled? The system must be able to perform accounting on the measured energy based on a pre-specified accounting model. The system must be transparent, decentralized, and secure. Anyone in the network should be able to verify the validity transactions. Finally, the system needs to scalable at reasonable cost.
\end{enumerate}

\section{Scope}
The Master Thesis explores the fundamental terms needed to understand blockchain terminology. The contributions to scalability are limited to optimizing smart contracts. Other scalability solutions are mentioned but in depth analysis is considered out of scope. On security we limit ourselves to the industry's best practices, as per the literature and utilize a tool provided by an auditing firm.

The technology used includes but is not limited to the Solidity programming language\footnote{Ethereum's most popular language for writing smart contracts}, the Truffle Framework for streamlining the compilation, testing and deployment process, Javascript for unit testing the smart contracts and the Python programming language for automating tasks, performing plotting and analyzing of data.

\section{Relative Work}
Techniques which illustrate more efficient smart contracts by storing less data on a blockchain are described in~\cite{stateless}. \cite{DBLP:journals/corr/ChenLLZ17} makes it clear that compiler optimizations in smart contracts still need improvements in order to avoid unnecessary expenses. On network level scalability there are various approaches such as executing transactions `off-chain'\footnote{An exchange of cryptographically signed messages that does not happen on a blockchain.}~\cite{raiden, funfair, counterfactual} and use a blockchain only for the final settling of a series of transactions. Other approaches exist which involve creating `sidechains' which can be used to offload the computational effort from the `mainchain' \cite{sidechains, loom, cosmos, plasmacash, plasma}. Finally, another approach to achieving scalability is via permissioned blockchains which trade decentralization and transparency for efficiency~\cite{hyperledger, Vukolic:2017:RPB:3055518.3055526}.

Extensive analyses have been performed on the security of blockchains as networks~\cite{Gervais:2016:SPP:2976749.2978341, cryptoeprint:2018:236} and specifically on the security of Ethereum smart contracts~\cite{Atzei:2017:SAE:3080353.3080363} which have proven to be insufficiently secure for the amounts of funds that they hold. As a result, tools that are able to analyze both source code and compiled bytecode for vulnerabilities have been developed~\cite{Luu:2016:MSC:2976749.2978309, mythril, echidna, smartcheck, securify, zeus}. A recent study \cite{greedyprodigal} illustrates how smart contracts that can freeze or cause loss of funds can be detected \cite{maian}.

Utilizing blockchain for Internet of Things is explored in \cite{iot, integrationiot}, while a model for billing and accounting with smart contracts is proposed in~\cite{billaccount}. Energy market use-cases are being piloted by~\cite{gridplus, powerledger} and prototypes are being tested such as \cite{brooklyn, DBLP:journals/ife/MengelkampNBDW18}

\section{Outline}

In Chapter~\ref{ch:basics} introduces terminology required for understanding blockchain fundamentals. We then proceed to explain how Ethereum works at a lower level, along with the programming techniques and tooling necessary to author smart contracts.

In Chapter~\ref{ch:scalability} we refer to the scalability issues that plague today's blockchain systems and provide a brief description of the known possible solutions that can be implemented to solve these problems. We make a contribution on scaling smart contracts which allows for more optimized writes to the Ethereum blockchain.

In Chapter~\ref{ch:security} we go over the 

In Chapter~\ref{ch:implementation} we implement the thing

In Chapter~\ref{ch:results} we explain our results from the thing

We conclude at~\ref{ch:conclusion}, summarize and give future work indications. 