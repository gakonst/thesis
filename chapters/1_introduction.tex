\chapter{Introduction}

\section{Motivation}
In 2009 Satoshi Nakamoto published the Bitcoin whitepaper~\cite{bitcoin} where he described `a purely peer-to-peer version of electronic cash', which `would allow online payments to be sent directly from one party to another without going through a financial institution'.

Bitcoin was primarily used for fast and low-cost financial transactions, which were routed without any bank interference. It was soon realized that its underlying technology, blockchain, could be used for more than transferring financial value. A blockchain is a database that can be shared between a group of non-trusting individuals, without needing a central party to maintain the state of the database. The data in a blockchain is transparent and secured via cryptography. As more advanced blockchain platforms were built on top of Bitcoin~\cite{colored}, in the end of 2014 a blockchain platform which was capable of executing smart contracts was released, called Ethereum~\cite{vitalik}. A smart contract, first referred to as a term by Szabo in \cite{szabo}, is software that is executed on a blockchain and can be used as a framework for secure and trustless computation.

Smart contracts became very popular because they enabled the concept of Decentralized Applications (DApps). Use cases include crowdfunding through Initial Coin Offerings (ICO), decentralized exchanges, where order matching and settling does not require a central operator, and decentralized autonomous organizations, where token shareholders can vote on proposals. Smart contracts operate as code with the potential to hold large amounts of funds for a particular use case. Past events have shown that security is very important as large financial amounts have been stolen from smart contracts. Finally, due to the distributed nature of blockchains there are challenges towards achieving scalability and high transaction throughput, which traditional centralized payment processors or server architectures provide.

\section{Research Topics}
We aim to explore the following research topics in this thesis:

\begin{enumerate}\label{statement}
    \item Blockchain architectures are hard to scale because of the need for decentralization while maintaining security. In particular, Ethereum's smart contracts allow more complex state transitions which leads us to research ways on how can the scalability of smart contracts be improved.
    \item Due to potentially large financial amounts being held in smart contracts and given the precedent of funds being stolen, we research how smart contract security be improved and how can proper auditing be streamlined in the development process.
    \item We explore how can a system that is able to measure and bill the energy consumed by a set of energy meters be modeled through the use of the Ethereum blockchain. The system must be able to perform accounting on the measured energy based on a pre-specified accounting model. The system must be transparent, decentralized, and secure. Anyone in the network should be able to verify the validity transactions. Finally, the system needs to scalable at reasonable cost.
\end{enumerate}

\section{Scope}
We examine important quality aspects of DApps like scalability and security and apply the findings in a pilot use case of energy metering and billing using Solidity smart contracts. 


On security we limit ourselves to the industry's best practices, as per the literature and evaluate performance of an automated security testing tool compared to other tools.

The architecture of the metering and billing model of energy consumed by the set of energy meters mentioned in Section \ref{statement} is implemented based on a specific use-case set by Honda Research and Development Germany, hereafter referred to as HREG, with which we collaborate for the purpose of the Master Thesis.

The technology used includes but is not limited to the Solidity programming language\footnote{Ethereum's most popular language for writing smart contracts}, the Truffle Framework for streamlining the compilation, testing and deployment process, Javascript for unit testing the smart contracts and the Python programming language for automating tasks and analyzing of data.

\section{Outline}

Chapter~\ref{ch:basics} introduces terminology required for understanding blockchain fundamentals. We then proceed to explain how Ethereum works at a lower level, along with the programming techniques and tooling necessary to author smart contracts.

In Chapter~\ref{ch:scalability}, we refer to the scalability issues that plague today's blockchain systems and provide a brief description of the known possible solutions that can be implemented to solve these problems. We make a contribution on scaling smart contracts which allows for more optimized writes to the Ethereum blockchain.

In Chapter~\ref{ch:security}, we go over the most significant security issues found in Ethereum and its smart contracts. Having understood these, we evaluate and augment the auditing tool \textit{Slither}'s ability to detect and identify vulnerabilities and compare it to taxonomy of tools described in \cite{tools}. Finally, we analyze smart contracts \textit{honeypots} as a technique used by adversaries to profit from existing and well known smart contract flaws.

In Chapter~\ref{ch:implementation}, we present ways in which smart contracts can address the energy market's inefficiencies and create a suite of smart contracts which are able to answer research question 3 from~\ref{statement}, while taking the lessons learned from Chapter~\ref{ch:scalability} and Chapter~\ref{ch:security} into account.

In Chapter~\ref{ch:results}, we evaluate the performance and the extent at which the smart contracts from Chapter \ref{ch:implementation} achieve their goal. 

In Chapter~\ref{ch:conclusion}, we reiterate and summarize on the findings from the previous chapters, highlight future work that can be done to further improve both the design of the described smart contracts but also the security and scalability of Ethereum.

\section{Writing Conventions}
Limited code snippets can be found across the document when needed for explanatory reasons, full code with explanations can be found in the Appendices and in the thesis' accompanying GitHub repository\@.