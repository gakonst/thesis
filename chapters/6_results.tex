\chapter{Evaluation of Implementation and Results}\label{ch:results}

We have described an ecosystem of smart contracts that combined with their corresponding client scripts can perform metering and billing of energy with complex business logic. We proceed to evaluate the performance of the implementation described in Chapter \ref{ch:implementation}, and explain how the findings from Chapter \ref{ch:scalability} and Chapter \ref{ch:security} are applied to enable better scalability and robustness in the contract design.

\section{Scalability}
On scalability, we applied Method 3 (\ref{method3}) and followed the rules from Section \ref{gas-costs} to save gas. More specifically:

\begin{enumerate}
    \item The optimizer flag was activated during compilation
    \item Using libraries via Method 3 for storing meter readings and department values resulted in significant code reuse, and reduction of gas costs. Table \ref{table:compare-gas} illustrates the gas savings that the proposed method provides, for each function of the contracts.
    \item Any meter, department that is no longer used can be deactivated, freeing up storage in the smart contract.
    \item Any iterations over arrays are initializing the break condition outside of the loop, reducing the number of \texttt{SLOAD} operations done and thus saving gas.
    \item In all cases, \texttt{string} is avoided and \texttt{bytes32} is used. % If a representation was needed that was larger than 32 bytes would be needed, it would be rendered in the front end by matching the 32 bytes to a potential string. 
    \item The minimal amount of data needed is stored in the smart contract. The latest reading and timestamp values are needed for each meter and department, and each department additionally requires storing its hierarchical relationship with other departments, virtual meters and its delegates. The formula for each virtual meter (\ref{metering}) is stored locally because it would significantly increase gas costs to store an arbitrarily long \texttt{string} variable in the smart contract.
    \item Events are emitted for each action providing an inexpensive way to store the historical data of a meter or department. Emitting an event requires significantly less gas compared to storing a value in contract storage~\cite{events}. The disadvantage is that event data cannot be accessed by another smart contract which is acceptable in this case as historical data are only meant to be accessed directly by a client and not by a smart contract.
\end{enumerate}

\begin{table}[htb]
	\centering
	\caption{Gas costs comparison between Method 1 and Method 3 for storing meter readings and department energy consumptions, optimizer runs = 200}
	\vspace*{-1ex}
	\scriptsize
	\vspace{-1ex}
	\begin{tabular}{|c|c|c|c|}
		\hline
		\textbf{Contract} & \textbf{Method 1} & \textbf{Method 3} & \textbf{Improvement} \\ \hline
		MeterDB::deployment		      &    793279 &   637403 & 20\% \\
		AccountingDB::deployment      &    1691164 &   1088253   & 36\%\\
		MeterDB::ping 			      &    57844 &   31422   & 46\% \\
        AccountingDB::billPower       &    82025 &   28250   & 65\%\\
        AccountingDB::fixedSplit      &    98126 &   47260   & 51\%\\
        AccountingDB::headcountSplit  &    231305 &   55024   & 76\%\\
		\hline
	\end{tabular}
	\vspace{-2ex}
	\label{table:compare-gas}
\end{table}

\section{Security}

Having studied and understood past literature as well as attacks on smart contracts, the implementation is developed so that none of the known attacks can be reproduced. In Section \ref{slither} we evaluated that Slither can detect a wide variety of vulnerabilities, and can outperform at most cases the other available security auditing tools. Auditing our smart contracts through Slither showed no true positives which, given the past performance of the tool, gives us a high degree of confidence with regard to the robustness of the developed smart contracts against known attacks.

We utilized and applied an access control list to the implemented smart contracts to enforce role based access control. Roles are allowed to access only the functions of a smart contract that they absolutely must have access to, following the security principle of least privilege. Each user is then granted a role, which provides a portable method to revoke and modify the permissions of a user. 

As smart contract ecosystems become more complex and the number of participating parties increase, ways to enforce access control are going to be increasingly important. Our approach allows for a central directory of permissions to exist, which makes it efficient to modify the permissions of each actor on a per-contract basis. 

\section{Complete overview}
The final implementation was tested with a setup of 48 smart meters, 44 virtual meters and 20 departments. The simulation was conducted on a Dell Laptop and consumed less than 10\% of the laptop's CPU and RAM resources. The deployment of the smart contracts was firstly tested on a local ganache\footnote{Software which allows setting up a local private blockchain with funded accounts and instant block confirmation times, shown in Figure \ref{fig:ganache}} testnet instance, then on a locally deployed testnet of 2 geth instances, where 1 was acting as a node and another as a miner, and finally on the Ropsten public testnet. Due to company requirements, there was no deployment to a public network, however since deployment to Ropsten was successful, it is expected that deployment to the mainnet will be successful as well.

Finally, we analyze the expected gas costs for operating the metering and billing logic of the developed smart contracts on the Ethereum mainnet for a year:

\begin{enumerate}
    \item \textbf{Meters:} The monitoring server is updated every 12 minutes, however since the change in values is very small, we poll the monitoring server and update the values of the smart meters every 2 hours\footnote{The selected frequency was a company requirement, as more frequent pings would increase cost and provide no significant advantage in granularity}. As a result, the operating cost for pinging the values of 48 meters for a year\footnote{The chosen gas price is $1 GWei = 10^{-9} ether$ since the operations are not time critical. When there is a need for fast confirmations the price can be set higher.} is: 48 meters * 31422 gas * 0.5 readings / hour * 24 hours * 365 days * 1 Gwei/gas  = 6.6 ether, which reduces to each meter needing 0.1375 ether in order to be able to interact with the smart contract at the specified frequency for a year. 
    \item \textbf{Virtual Meters:} Pushing the values of Virtual Meters is required to be done twice every month and so the total cost of running 44 virtual meters is: 44 virtual meters * 31422 gas * 2 readings / month * 12 months * 1 Gwei/gas  = 0.03 ether, which reduces to each virtual meter needing 0.0007 ether to interact with the smart contract at the specified frequency for a year. 
    \item \textbf{Billing:} Calculating the consumption of each business unit and performing cost forwarding for internal accounting is required to be done once every month. It was simulated that a full billing cycle of pulling values from virtual meters, updating each department's consumption and delegating it according to the business logic costs X ether per month, or X ether per year.  % TODO GET VALUES
\end{enumerate}
