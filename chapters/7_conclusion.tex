\chapter{Conclusion}\label{ch:conclusion}

\section{Future Work}
The main issue with our curernt implementation is that instead of having a direct push from each meter (or any IoT device) to the blockchain, we need to pull the data from the aforementioned monitoring server, and then push it again. This introduces latencies and single points of failure, however, this was done due to our corporate setup. An improvement would be to setup a microcontroller on each that would be executing a binary that pings readings to the blockchain. even better, run a node on each IOt device, however requires too much power, maybe in the far future. We explore Ethereum platform due to its abundance in developers and stay in it. There are other smart contract platforms however they lack developer tools, are not battle tested and are potentially centralized. As there is a bigger issue with scaling, the whole infrastructure could be transferred to a permissioned in-house blockchain, however we wanted to stay within the scope of keeping it as transparent as possible.

FIN.