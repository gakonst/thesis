\chapter{Blockchain and the Energy Market}

Having discussed how Ethereum works, and explained techniques to improve smart contracts' scalability and security, we proceed to discuss the topic of making energy markets more transparent and efficient by utilizing smart contracts. The use-case we describe can be used as a starting point for better tracking of energy usage inside a company, allowing better prediction of future needs.  

\section{Energy Market inefficiencies}

The world is gradually shifting from nuclear and fossil fuels to Renewable Energy Sources (RES). RES have been taking a larger share of Germany's gross energy production and this has created a 
Germany is on a rollout plan of installing smart meters in every household which incorporates RES.  

The barrier to entry to become an energy producer\footnote{By installing solar panels for example} 
% Chart for renewables: https://www.cleanenergywire.org/factsheets/germanys-energy-consumption-and-power-mix-charts

In its current state, most consumers do not know what they are paying Business level take long time and are 

Price of energy, consumer does not know always what they pay, or what they gain from their renewables
 
\section{Advantages of Blockchain}

\subsection{Peer 2 Peer}
In the most general cases, blockchains have the ability to provide transparency and immutability. When talking about energy and transparency, full history of meter readings, price calculation, billing of inhouse energy departments. This can be extended for EV car payment microtransactions and so on.

\subsection{User Owned Data}
% The blockchain is a particularly interesting technology for decentralized processes that require large networks and trust relationships between all parties. Therefore, it offers great benefits to the power & utilities market, with its large networks of power and utilities companies, (maintenance) subtractors, (local) suppliers and end users.

% Smart meters Still, there are other applications that are ready for use in the near future. Smart meters for instance have already entered many homes. Up until now, sharing data through smart meters was a threat to the privacy of the owner of the meter. Again, the blockchain offers a potential solution. It can provide the accurate data to the supplier without requiring a direct link to the meter of specific users. When needed, the owner of the smart meter can prove to the supplier that the data are his, using his private key, and the cryptographic security of the blockchain proves that the information is accurate and hasn’t been tampered with. This puts the owner in control of his own data.

\section{Our Use-case}\label{usecase}
In collaboration with Honda R\&D Germany, we create a pilot suite of smart contracts for in-house use in order to track and bill the consumed energy of the company's headquarters as measured by a set of smart meters. 

Describe meters, billing and so ona
The purpose is to serve as a means of tracking the energy consumed by the company's smart meters and ensuring the data's validity and existence in a smart contract. In addition, due to the complex structure of the company, every smart meter's consumption can contribute with different coefficients to the total energy consumption of the rooms in a building. As a result, the developed smart contract are able to track the energy consumption of each room and assign it to a higher-order 