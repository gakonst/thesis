\chapter{Scalability} \label{apx:scalability}

\section{Advantages of the Optimizer}

The Solidity compiler optimizer operates on assembly. It is able to analyze the compiled code and break it down in blocks, where each instruction gets analyzed~\cite{docsmisc}. It tries to simplify expressions which can be evaluated at runtime, and `pack' items that are less than 32 bytes long into a single storage slot, according to the following rules:

\begin{enumerate}
  \item The first item in a storage slot is stored lower-order aligned.
  \item Elementary types use only that many bytes that are necessary to store them.
  \item If an elementary type does not fit the remaining part of a storage slot, it is moved to the next storage slot.
  \item Structs and array data always start a new slot and occupy whole slots (but items inside a struct or array are packed tightly according to these rules).
\end{enumerate}

Variable packing is very efficient because writing to storage which has been already written to before by \text{SSTORE} requires only 5000 gas instead of 20000 (\ref{table:opcode_table}). The execution cost for \texttt{set()} in Figure \ref{fig:optimizer:a} is ~60000\footnote{The gas costs were obtained by calling the function in Remix IDE, \url{http://remix.ethereum.org/}} because it initializes 3 storage slots, compared to \ref{fig:optimizer:b} where it costs ~45000 because it initializes 2 storage slots\footnote{The \texttt{uint128} items get packed in one storage slot.}. By using the optimizer, the third \texttt{SSTORE} is removed as shown in Figure \ref{fig:optimizer:c} because the compiler can set both values with 1 \texttt{SSTORE} command in the same storage slot. As a result, the gas cost for calling \texttt{set()} with the optimizer in \ref{fig:optimizer:b} is ~40000 and requires only 2 \texttt{SSTORE} calls, compared to Figure \ref{fig:optimizer:a} where the optimizer is unable to provide any additional savings.

\begin{figure}[htb]
  \begin{subfigure}[b]{\textwidth}
    \centering
    \lstinputlisting[language=Solidity]{contracts/CannotPack.sol}
    \label{fig:optimizer:a}
  \end{subfigure}

  \begin{subfigure}[b]{\textwidth}
    \centering
    \lstinputlisting[language=Solidity]{contracts/CanPack.sol}
    \label{fig:optimizer:b}
  \end{subfigure}

  \begin{subfigure}[b]{\textwidth}
    \centering
    \lstinputlisting{code/solc.txt}
    \label{fig:optimizer:c}
  \end{subfigure}

  \caption{The optimizer saves a call to \texttt{SSTORE} when it operates on packed storage variables.} 
  \label{fig:optimizer}
\end{figure}

\section{Using Libraries in Smart Contracts}

Libraries are contracts which cannot hold ether and do not have storage. Using libraries results in gas savings and code reuse. According to the Solidity Documentation, `Libraries can be seen as implicit base contracts of the contracts that use them'. The functions of deployed Librares are called by using the opcode \texttt{DELEGATECALL}, which allows a function to be executed in the context of the caller. This results in contracts being able to delegate complex computation on an already deployed library's functionality, without having to deploy the library code. Libraries can be used to modify their associated contract's variables. Utilizing the \texttt{using} syntax, a function in a library can be called as a method of its first parameter as shown in Figure \ref{fig:usingfor}.

\begin{figure}[htb]
    \centering
    \lstinputlisting[language=Solidity]{contracts/Counter.sol}
    \caption{Example of using the \texttt{using X for Y} syntax to create a \texttt{Counter} datatype~\cite{library}}
    \label{fig:usingfor}
\end{figure}

Our implementation of the library for saving gas requires firstly defining the layout of variables in storage, as shown in Table \ref{table:meterlayout} for a Meter. 

\begin{table}[htb]
	\centering
	\vspace*{-1ex}
	\scriptsize
	\vspace{-1ex}
	\caption{Layout of variables required to store a meter in a 256 bit word}
	\begin{tabular}{|c|c|c|c|}
        \hline
        \textbf{Type} &     \textbf{Name}    & \textbf{Index} & \textbf{Bits} \\ \hline 
            bool      &     active           &    0           &      1  \\
            bytes8    &     id               &    1           &      64 \\
            uint48    &     power            &    65          &      48 \\
            uint32    &     timestamp        &    113         &      32 \\
            uint48    &     lastPower        &    145         &      48 \\
            uint32    &     lastTimestamp    &    193         &      32 \\
        \hline
    \end{tabular}
    \label{table:meterlayout}
\end{table}


After that, we define a \texttt{mask} variable for each size of the datatypes we use and a variable that indicates how many shifts must be done to set or retrieve a variable, as shown in Figure \ref{fig:masking:a}. For example, setting \texttt{lastPower} requires shifting left 145 bits and applying a 48 bit mask, shown in Figure \ref{fig:masking:b}. The exposed API for setting the value is abstracted from the end user as shown in Figure \ref{fig:masking:c}.

\begin{figure}[H]
  \begin{subfigure}[b]{\textwidth}
    \centering
    \lstinputlisting[linerange={20-20,27-27}, language=Solidity]{contracts/MeterMaskingLib.sol}
   \caption{Mask and shift offsets for lastPower}
    \label{fig:masking:b}
  \end{subfigure}

  \begin{subfigure}[b]{\textwidth}
    \centering
    \lstinputlisting[firstline=32, lastline=40, language=Solidity]{contracts/MeterMaskingLib.sol}
    \caption{Getting and setting a property}
    \label{fig:masking:a}
  \end{subfigure}

  \begin{subfigure}[b]{\textwidth}
    \centering
    \lstinputlisting[linerange={63-68,76-81}, language=Solidity]{contracts/MeterMaskingLib.sol}
    \caption{Getting and Setting lastPower API}
    \label{fig:masking:c}
  \end{subfigure}

  \caption{API for setting and retrieving values by using a library}
  \label{apx:scalability:lib}
\end{figure}

\section{Comparing Gas Costs across Implementations}

In Section \ref{ch:scalability:results} we made gas comparisons between the 3 developed methods, in order to determine which was the most efficient in terms of gas. The process followed for each implementation was:

\begin{enumerate}
  \item Compile the contract with $N$ optimizer runs, where $N$ in $[0, 1, 100, 500, 500000]$ 
  \item Deploy the compiled ABI and Bytecode to a local testnet\footnote{We used ganache as a testnet.}
  \item Get gas costs for deployment.
  \item Execute each of the contract functions\footnote{The functions were called once since the gas costs are deterministically the same for all calls}.
  \item Get gas costs for each function call.
\end{enumerate}

For that, we implemented a set of Python scripts responsible for automating the process. Specifically, we created:
\begin{enumerate}
  \item A \texttt{getWeb3} script which allows connecting to an Ethereum node by providing an endpoint with the syntax \texttt{w3 = getWeb3(<endpoint>)}.
  \item A \texttt{framework} script which exports a \texttt{Contract} class that can compile and deploy a smart contract when given its source code. It also provides a \texttt{call} method which allows calling a smart contract function and retrieve its gas usage. 
  \item A comparison script which utilizes the \texttt{framework} and the \texttt{pandas} library to create a dataframe with the gas costs for each function of the supplied smart contracts, for the specified optimizer runs.
\end{enumerate}